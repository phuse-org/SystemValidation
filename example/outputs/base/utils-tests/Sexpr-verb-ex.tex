\documentclass[11pt,a4paper]{article}

%% *** automatically switched by 'make' ( ./Makefile ) --- CARE!! in changing
   % Exercise mode
%\SweaveOpts{echo=TRUE,eval=TRUE,results=verbatim} % Solution mode
%% other Sweave options


\newif\ifSolution
\Solutiontrue% if solution
\Solutionfalse%if exercise
%
\ifSolution\newcommand{\commentSol}[1]{#1}
\else      \newcommand{\commentSol}[1]{}
\fi
\newcommand{\T}[1]{\texttt{#1}}

\usepackage{Sweave}
\begin{document}


We work with the data set \T{airquality} which is part of R....
You can address it simply by \T{airquality}. Use \T{?airquality} to read about the
meaning of the variables contained in the dataset.

Get a summary of the data,
\begin{Schunk}
\begin{Sinput}
R> summary(airquality)
\end{Sinput}
\end{Schunk}

\commentSol{The data set contains \verb#<<nrow(airquality)>># observations. The data is
complete for all but the first two variables \T{Ozone}, \T{Solar.R},
which contain \verb#<<sum(is.na(airquality[,1]))>># and
\verb#<<sum(is.na(airquality[,2]))>># missing values, respectively.
}

The above works in solution mode,
but in exercise mode, the $\backslash$Sexpr results are put out verbatim,
unfortunately using $\backslash$\verb|verb{bla bla{|  and the left brace
\emph{really} messes up the $\backslash$commentSol\verb|{..}| command...

\end{document}
